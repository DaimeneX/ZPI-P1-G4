\documentclass[12pt,a4paper]{article}
\usepackage[polish]{babel}
\usepackage[T1]{fontenc}
\usepackage[utf8x]{inputenc}
\usepackage{hyperref}
\usepackage{url}
\usepackage{graphicx}

\addtolength{\hoffset}{-1.5cm}
\addtolength{\marginparwidth}{-1.5cm}
\addtolength{\textwidth}{3cm}
\addtolength{\voffset}{-1cm}
\addtolength{\textheight}{2.5cm}
\setlength{\topmargin}{0cm}
\setlength{\headheight}{0cm}

\begin{document}

\title{Dokumentacja projektu ZPI}
\author{Dąbrowski Daniel\\ Gródek Damian\\ Jargut Piotr\\ Kocjan Katarzyna\\ Uryga Kamil}
\date{\today}

\maketitle

\begin{abstract}
Zarys dokumentacji projektowej stanowiący podstawę do realizacji projektu \\ zespołowego. 
\end{abstract}



\newpage

\tableofcontents
\listoftables
\listoffigures



\newpage

\section{Tytuł: System wspomagający szybkie parkowanie}

\section{Nazwa robocza: Parkingowy}

\section{Cel}
\begin{quotation}
Stworzenie systemu wspomagającego znalezienie jak najszybciej \\użytkownikowi aplikacji wolne miejsce na trzypoziomowym parkingu.
\end{quotation}

\section{Zakres}

\subsection{Analiza wymagań}
\begin{itemize}
\item Określenie optymalnego położenia emiterów beacon na podstawie planu parkingu \\i zasięgu emiterów;
\item Instalacja serwera;
\item Stworzenie bazy danych i szkieletu aplikacji;
\item Implementacja systemu zarządzania wolnymi  miejscami;
\item Instalacja Acesspointów wifi mesh z funkcją emitera beacon;
\item Implementacja funkcji lokalizacji telefonu na terenie parkingu;
\item Implementacja systemu komunikacji pomiędzy czujnikami zajętości miejsca a serwerem za pośrednictwem protokołu MQTT;
\item Implementacja funkcji nawigacji do wolnego miejsca parkingowego;
\item Rozmieszczenie czujników zajętości miejsca;
\item Implementacja systemu komunikacji pomiędzy szlabanami a serwerem;
\item Postawienie szlabanów;
\item Test;
\item Gotowy system.
\end{itemize}

\newpage

\subsection{Wymagania funkcjonalne i niefunkcjonalne}
{\large \bf Wymagania funkcjonalne :}
\begin{itemize}
\item Sprawdzenia ilości wolnych miejsc;
\item Sprawdzenia ilości zajętych miejsc;
\item Znalezienia najbliższego wolnego miejsca parkingowego;
\item Poinformowaniu o ewentualnym braku miejsc na danym piętrze;
\item Poinformowaniu o zwolnieniu najbliższego wolnego miejsca;
\item Aplikacja doprowadza na najbliższe miejsce parkingowe za pomocą systemu lokalizacji opartego na beacon.
\end{itemize}
{\large \bf Wymagania niefunkcjonalne:}
\begin{itemize}
\item Aplikacja powinna być darmowa;
\item Aplikacja powinna mieć intuicyjny i przejrzysty interfejs;
\item Aplikacja powinna być kompatybilna z większością urządzeń aby mogło używać jej jak najwięcej użytkowników.
\end{itemize}

\subsection{Diagram przypadków użycia i diagram przepływu (opcjonalny)}
\begin{figure}[htb!p]
\includegraphics[width=0.9\textwidth]{Untitled Diagram.png}
\caption{Diagram przypadków użycia}
\end{figure}

\subsection{Dobór technologii}
\begin{itemize}
\item Emitery beacon pracujące na protokole BT low energy(Access point’y zintegrowane \\z emiterami Beacon);
\item Magnetyczne czujniki zajętości miejsca (wifi,MQTT);
\item Windows Server;
\item Program RabitMQTT, Mosquito;
\item Android Studio(język JAVA);
\item C\# (opcjonalne);
\item Unity (opcjonalne).
\end{itemize}













%\section{Scenariusze}
%(tytuł, numer, aktorzy, stan wejścia (warunki + dane), przebieg scenariusza, wynik, scenariusz alternatywny, jeśli istnieje)
%\section{Estymacja czasowa}
%(poszczególnych zadań jak i określenie wymagań MVP oraz terminu końcowego oddania)
%\section{Implementacja}
%\section{Testy i ich wyniki}
%\section{Podsumowanie i bilans}
%(MVP vs rzeczywistość)
\end{document}